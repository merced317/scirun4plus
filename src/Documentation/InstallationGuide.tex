\documentclass[fleqn,12pt,openany]{book}

% These two need to be set before including scirun style package
\title{SCIRun Installation Guide}
\author{Jeroen Stinstra}

% INCLUDE SCI STYLE DOCUMENT
\usepackage{scirun}

\begin{document}

% CREATE TITLE PAGE --------------------------------------------------
\maketitle

% CHAPTERS ---------------------------------------------------------------

\chapter{Installation Guide}


\section{Getting Started}

SCIRun version \SCIRunVersion is currently available as a binary download for Windows XP/Vista and for Mac OS X 10.5 and better.
For all other platforms one currently needs to build SCIRun from source.
We recommend to only use the source distribution for building SCIRun on Linux or for those users who want to contribute to the code.
This installation guide deals with both installing the binary distribution and the source distribution.

\section{Distribution Model}

SCIRun is currently being distributed through two channels:
\begin{itemize}
\item code and binaries can be downloaded from our \href{http://software.sci.utah.edu}{website}
\item the SCIRun code can be checked out through our subversion repository (see section \hyperref[dev]{\ref{dev} downloading for development})
\end{itemize}

At all times two versions of SCIRun are available:
\begin{itemize}
\item the release version of SCIRun, which is currently version \SCIRunVersion.
\item the development version of SCIRun, which is currently version 4.4.
\end{itemize}

The release version is intended to be the most stable version in which will only be updated with bug fixes.
The development version contains the latest features, but may not be entirely stable as new code is added frequently.
As both versions are continually updated, up-to-date versions can be downloaded from our website. 

Both binary and source files are generated on a weekly basis and downloads are encoded by date to ensure
the user will always download the latest version.
Hence if a bug report is submitted and a solution is found, new versions will be available quickly through the website.

To check which version of SCIRun is being used, one can always look in the top right corner of the main SCIRun window:
it will list the SCIRun version and build revision number of that specific build. 

\section{System Requirements}

\subsection{Windows}

For Windows we highly recommended using the binary downloads as building SCIRun from source on these platforms is more complicated.

In order to make use of the volume rendering capabilities within SCIRun, it is recommended to use an NVIDIA or ATI graphics card.
As not all Windows installations come with the proper drivers installed for the graphics cards, we recommended upgrading the graphics card drivers before installing SCIRun.
The latest graphics card drivers can generally be downloaded from the card manufacturer's website.

\subsection{OS X}

For OS X binary downloads are available for OS X 10.5 and up.
We do not support any older version of OS X, and the current version of SCIRun fails to build on these older operating systems.
Currently both PowerPC and Intel-based workstations are supported for OS X 10.5 and the binary downloads are built as universals that should work on any Macintosh machine that supports OS X 10.5 or 10.6.

One needs to have an NVIDIA or ATI graphics card to make use of the volume rendering capabilities in SCIRun.
Laptops that rely on an Intel graphics card generally do not support the functionality that is needed for volume
rendering, and hence these components will not be available on these laptops.
SCIRun will work on these laptops but certain functionality in its kernel will be switched off. 

\subsection{Linux}

We currently test our SCIRun distribution on Fedora, SuSE and Ubuntu.
However SCIRun will probably work fine on other Linux distribution as well.
Currently SCIRun has to be build from source and it is recommended to have a GCC version of 3.4 or higher.

For Linux platforms we recommend against the use of ATI graphics cards as the graphics drivers are not compatible with SCIRun 4.x.
For doing any volume rendering on Linux we recommend the use of a NVIDIA graphics card.

As newer Linux distributions become available frequently, we will upgrade our code to be compatible with the latest
compilers and features, however we cannot guarantee that our code will work as soon as a new compiler/linux version
becomes available. 

\subsection{Compilers}

Visual Studio 2008 or the Visual Studio 2008 Express Edition are required to compile SCIRun on Windows.
We no longer support older compilers such as Visual Studio 2003 and Visual Studio 2005.

To compile SCIRun on OS X or Linux, we require the GCC/G++ compiler version 3.4 or newer.
As we cannot test any older versions, we cannot guarantee that the code base will compile on those compilers.
The SCIRun version 4.0 will compile on older versions, however more recent versions will require newer compilers.

\section{Installing the binary version}

Download installers from the
\href{http://software.sci.utah.edu}{SCI software portal}
and the SCIRun product link.

\subsection{Windows}

Download the win32 or win64 installer application and launch the installer.
The win64 installer contains the full 64bit version that allows the user to generate larger datasets and simulations.
The installer will guide the user through the process of installing SCIRun.
After SCIRun has been installed, run SCIRun from the start menu. 

\subsection{OS X}

OS X users can download a 32bit or a 64bit version of SCIRun depending on what their processor will support.
The latter version is a full 64bit build that allows the user to generate larger datasets and simulations.
Double click on the DMG file to mount the disk image and then open the disk image and drag the SCIRun Application to /Applications or to any other desired location.
Double click on the application icon to launch SCIRun.

\section{Installing the source distribution}

The instructions for installing the Linux and source archives are essentially the same.
Download the tar archive from the SCIRun download site and unpack.
Use the \texttt{build.sh} script to build SCIRun; the script will download and build the
\href{http://www.cmake.org}{CMake} build system (the \texttt{wget} or \texttt{curl} command line tools are needed)
if the CMake tools are not found in \textbf{PATH}.
Usage information is available:
\begin{verbatim}
./build.sh --help
\end{verbatim}

SCIRun can be built from source on Windows by configuring the SCIRun build with CMake (using the
CMake GUI is strongly recommended) to generate Visual Studio 2008 project files.
Both 32 and 64 bit architectures are supported.
More detailed information about how to build SCIRun on Windows can be found in the Developer Guide.

\section{Downloading source from Subversion for development}\label{dev}

Anonymous checkout is supported for read-only access to the SCIRun Subversion repository.
Developers with a SCI account can use their SCI account to authenticate with the Subversion server.
Full access information is available from the project
\href{https://code.sci.utah.edu/gf/project/cibc/scmsvn/?action=AccessInfo}{Subversion portal}.
Anonymous checkout is also supported in Windows Subversion clients such as TortoiseSVN.

\begin{verbatim}
svn checkout --username anonymous \
  https://code.sci.utah.edu/svn/cibc/cibc/trunk SCIRun
\end{verbatim}

\noindent This will include SCIRun data from Subversion as well as the source code.
To only download SCIRun source code:

\begin{verbatim}
svn checkout --username anonymous \
  https://code.sci.utah.edu/svn/cibc/cibc/trunk/SCIRun SCIRun_source
\end{verbatim}

SCIRun data can be obtained separately from Subversion.
Full access information is available from the project
\href{https://code.sci.utah.edu/gf/project/CIBCData/scmsvn/?action=AccessInfo}{Subversion portal}.

\begin{verbatim}
svn checkout --username anonymous \
  https://code.sci.utah.edu/svn/CIBCData/SCIRunData
\end{verbatim}

\end{document}
% ------------------------------------------------------------------------
